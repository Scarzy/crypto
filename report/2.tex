\section{Hamming Codes}

\subsection{a}
The first steps taken were to determine if the functions were reducible or not, as reducible functions can not be generator functions.

\subsubsection{i}
$
f_{1}(x) = x^{6} + x^{3} + x^{2} + x + 1 \\
f_{1}(0) = 0^{6} + 0^{3} + 0^{2} + 0 + 1 \\
f_{1}(0) = 1	\\
f_{1}(1) = 1^{6} + 1^{3} + 1^{2} + 1 + 1 \\
f_{1}(1) = 0 + 1 + 0 + 1 + 1	\\
f_{1}(1) = 1	\\
$

This function is irreducible, as both $f_{1}(0)$ and $f_{1}(1)$ equal 1.

\subsubsection{ii}
$
f_{2}(x) = x^{6} + x^{5} + x^{4} + x^{3} + 1 	\\
f_{2}(0) = 0^{6} + 0^{5} + 0^{4} + 0^{3} + 1 	\\
f_{2}(0) = 1	\\
f_{2}(1) = 1^{6} + 1^{5} + 1^{4} + 1^{3} + 1 	\\
f_{2}(1) = 0 + 1 + 0 + 1 + 1 	\\
f_{2}(1) = 1 	\\
$

This function is irreducible, as both $f_{2}(0)$ and $f_{2}(1)$ equal 1.

\subsubsection{iii}
$
f_{3}(x) = x^{6} + x^{5} + x 	\\
f_{3}(0) = 0^{6} + 0^{5} + 0 	\\
f_{3}(0) = 0	\\
f_{3}(1) = 1^{6} + 1^{5} + 1 	\\
f_{3}(1) = 0 + 1 + 1 	\\
f_{3}(1) = 0 	\\
$

This function is irreducible, as both $f_{2}(0)$ and $f_{2}(1)$ equal 1.??

\subsubsection{iv}
$
f_{4}(x) = x^{6} + x^{5} + x^{4} + x^{3} + x^{2} + x + 1 	\\
f_{4}(0) = 0^{6} + 0^{5} + 0^{4} + 0^{3} + 0^{2} + 0 + 1 	\\
f_{4}(0) = 1	\\
f_{4}(1) = 1^{6} + 1^{5} + 1^{4} + 1^{3} + 1^{2} + 1 + 1 	\\
f_{4}(1) = 0 + 1 + 0 + 1  + 0 + 1 + 1 	\\
f_{4}(1) = 0 	\\
$

This function is reducible, as both $f_{4}(0)$ and $f_{4}(1)$ are different.

\subsubsection{v}
$
f_{2}(x) = x^{6} + x + 1 	\\
f_{2}(0) = 0^{6} + 0 + 1 	\\
f_{2}(0) = 1	\\
f_{2}(1) = 1^{6} + 1 + 1 	\\
f_{2}(1) = 0 + 1 + 1 	\\
f_{2}(1) = 0 	\\
$

This function is reducible, as both $f_{5}(0)$ and $f_{5}(1)$ are different.

$f_{1}(x)$ is an appropriate generator polynomial, and can be used to generate the following sequence:
$
111111	\\
011111	\\
101111	\\
010111	\\
101011	\\
110101	\\
111010	\\
011101	\\
001110	\\
000111	\\
100011	\\
110001	\\
011000	\\
001100	\\
100110	\\
110011	\\
111001	\\
011100	\\
101110	\\
110111	\\
011011	\\
001101	\\
000110	\\
000011	\\
000001	\\
100000	\\
110000	\\
111000	\\
111100	\\
011110	\\
001111	\\
100111	\\
010011	\\
001001	\\
100100	\\
010010	\\
101001	\\
010100	\\
101010	\\
010101	\\
001010	\\
100101	\\
110010	\\
011001	\\
101100	\\
010110	\\
001011	\\
000101	\\
000010	\\
100001	\\
010000	\\
001000	\\
000100	\\
100010	\\
010001	\\
101000	\\
110100	\\
011010	\\
101101	\\
110110	\\
111011	\\
111101	\\
111110	\\
$

\subsection{b}
The first observation we make is that $n = 63$, which leads us to determine that \\ $m = log_{2}(n+1) \\ m = log_{2}(64) \\ m = 6$ \\
Which allows us to prove that \\ $k = 2^{m} - m - 1 \\ k = 2^{6} - 6 - 1 \\ k = 64 - 7 \\ k = 57$ \\
